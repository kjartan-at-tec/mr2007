\documentclass[border=20pt]{standalone}  
\usepackage[american,siunitx]{circuitikz}
\usetikzlibrary{arrows,shapes,calc,positioning}

\newcommand{\mymotor}[2] % #1 = name , #2 = rotation angle
{\draw[thick,rotate=#2] (#1) circle (10pt)
 node[]{$\mathsf M$} ++(-12pt,3pt)--++(0,-6pt) --++(2.5pt,0) ++(-2.8pt,6pt)-- ++(2.5pt,0pt);
\draw[thick,rotate=#2] (#1) ++(12pt,3pt)--++(0,-6pt) --++(-2.5pt,0) ++(2.8pt,6pt)-- ++(-2.5pt,0pt);
}

\begin{document}

\begin{circuitikz}
\draw (0,2) to [R=$R$,o-] (2,2) to[L=$L$] (4,2);
\draw (4,2) to[sV, color=white, name=M4] (4,0);
\mymotor{M4}{90}
\draw (0,0) to [short,o-] (4,0);

\draw[->] (-0.2,0.5) -- node [left] {$u(t)$} (-0.2, 1.5);
\draw[->] (3,0.2) -- node [above] {$i(t)$} (2,0.2);
\draw[->] (4.7,0.7) -- node [right] {$V_m(t)$} (4.7,1.3);
\end{circuitikz}

\end{document}
