% Created 2015-11-13 Fri 09:36
\documentclass{scrartcl}
\usepackage[utf8]{inputenc}
\usepackage[T1]{fontenc}
\usepackage{fixltx2e}
\usepackage{graphicx}
\usepackage{longtable}
\usepackage{float}
\usepackage{wrapfig}
\usepackage{soul}
\usepackage{textcomp}
\usepackage{marvosym}
\usepackage{wasysym}
\usepackage{latexsym}
\usepackage{amssymb}
\usepackage{hyperref}
\tolerance=1000
\usepackage[margin=18mm]{geometry}
\usepackage{amsmath}
\usepackage{graphicx}
\usepackage{subfigure}
\usepackage{parskip}
\usepackage{standalone}
\usepackage{tikz,pgf,pgfplots}
\usetikzlibrary{decorations.pathmorphing,patterns}
\usetikzlibrary{arrows,snakes,backgrounds,patterns,matrix,shapes,fit,calc,shadows,plotmarks,decorations.markings,datavisualization,datavisualization.formats.functions,intersections,external}
\usetikzlibrary{decorations.pathmorphing,patterns}
\pgfplotsset{compat=1.9}
\newcommand*{\mexp}[1]{\ensuremath{\mathrm{e}^{#1}}}
\newcommand*{\laplace}[1]{\ensuremath{\mathcal{L} \{#1\}}}
\newcommand*{\laplaceinv}[1]{\ensuremath{\mathcal{L}^{-1} \{#1\}}}
\newcommand*{\realpart}[1]{\ensuremath{\operatorname{Re}(#1)}}
\newcommand*{\impart}[1]{\ensuremath{\operatorname{Im}(#1)}}
\newcommand*{\vsp}[1]{\rule{0pt}{#1}}
\newcommand*{\tderiv}[1]{\ensuremath{\frac{d^{#1}}{dt^{n}}}}
\newcommand*{\bbm}{\begin{bmatrix}}
\newcommand*{\ebm}{\end{bmatrix}}
\newcommand*{\obsmatrix}{\mathcal{O}}
\newcommand*{\contrmatrix}{\mathcal{C}}
\newcommand*{\cwh}{\ensuremath{\cos \omega h}}
\newcommand*{\swh}{\ensuremath{\sin \omega h}}
\providecommand{\alert}[1]{\textbf{#1}}

\title{Computerized control - homework 2}
\author{Kjartan Halvorsen}
\date{Due 2015-09-03}
\hypersetup{
  pdfkeywords={},
  pdfsubject={},
  pdfcreator={Emacs Org-mode version 7.9.3f}}

\begin{document}

\maketitle


\section{Exercises}
\label{sec-1}
\subsection{Sample the continuous-time transfer function}
\label{sec-1-1}

   The harmonic oscillator from Homework 1
\begin{align*}
\dot{x} &= \begin{bmatrix} 0 & \omega\\-\omega & 0 \end{bmatrix} x + \begin{bmatrix}1\\0\end{bmatrix} u\\
y &= \begin{bmatrix} 1 & 0 \end{bmatrix} x.
\end{align*} 
has the transfer function 
\[ G(s) = C(sI-A)^{-1}B + D = \frac{s}{s^2 + \omega^2}. \]
Sampling the state space system with zero-order-hold gives the discrete-time state space system ($x(kh) = x(k)$)
\begin{align*}
x(k+1) &= \bbm \cos \omega h & \sin \omega h\\ -\sin \omega h & \cos \omega h \ebm x(k) + 
          \frac{1}{\omega} \bbm \sin \omega h \\ \cos \omega h - 1 \ebm u(k), \\
y(k) &= \bbm 1 & 0 \ebm x(k).
\end{align*}

\begin{enumerate}
\item \textbf{Compute the pulse-transfer function} for the discrete-time system from the state-space representation using the expression \[ H(z) = C(zI-\Phi)^{-1}\Gamma. \]
\item \textbf{Compute the pulse-transfer function} by sampling the transfer function $G(s)$.
\end{enumerate}
\subsection{Simulation of the continuous- and discrete-time harmonic oscillator}
\label{sec-1-2}
\subsubsection{Simulate step responses}
\label{sec-1-2-1}

Use matlab's control toolbox or the \href{http://python-control.sourceforge.net/}{python control module}  to simulate the systems. Use $\omega=1$. 

First, define the continuous-time system \texttt{sys\_c} and the sampled system \texttt{sys\_d} using the \texttt{ss} function. The example below uses the python control toolbox. Using the matlab control toolbox is very similar.

\begin{verbatim}
import numpy as np
import control.matlab as cm
import matplotlib.plot as plt

omega = 1.0
h = omega / 10

A = np.array([[0, omega], [-omega, 0]])
B = np.array([[1],[0]])
C = np.array([[1, 0]])
D = np.array([[0]])
sys_c = cm.ss(A,B,C,D)

wh = omega*h
F = np.array([[np.cos(wh), np.sin(wh)], [-np.sin(wh), np.cos(wh)]])
G = 1.0/omega* np.array([[np.sin(wh)], [np.cos(wh)-1]])
sys_d = cm.ss(F,G,C,D, h)

Tc = np.linspace(0,4/omega,200)
(yc,tc) = cm.step(sys_c, Tc)
Td = h*np.arange(40)
(yd,td) = cm.step(sys_d, Td)

plt.plot(tc,yc)
plt.plot(td,yd[0], '*')
\end{verbatim}

\textbf{Verify that the step response of the discrete-time system is equal to that of the continuous-time system at the sampling instants. Explain why this is so!}
\subsubsection{Sampling the system with help of the computer}
\label{sec-1-2-2}

Use the function \texttt{c2d} to sample your continuous-time system \texttt{sys\_c}. \textbf{Verify that you get the same discrete-time system as your} \texttt{sys\_d} \textbf{above}. \emph{Hint}: Look at the system matrices returned by \texttt{ssdata}. 
\subsubsection{Compute the discrete step response yourself}
\label{sec-1-2-3}

    Write some lines of code that solves the difference equation
    \begin{align*}
    x(k+1) &= \Phi x(k) + \Gamma u(k)\\
    y &= Cx(k)
    \end{align*}

given an initial state $x(0)=x_0$ and an input sequence $\{u(k)\}$. Use a step signal ($u(k)=1$) and verify that your solution is the same as when using the \texttt{step} function.
 

\end{document}
